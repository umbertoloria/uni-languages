\documentclass[]{article}
\usepackage[utf8]{inputenc}
\usepackage{comment}
\usepackage{mathtools}

\title{Appunti SFB 2023/2024}
\author{Umberto Loria}
\date{Ottobre 2023}

\begin{document}

\begin{titlepage}
\maketitle
\end{titlepage}

\tableofcontents{}
\newpage

\section{Alfabeti e linguaggi}

Un \textit{alfabeto} \mbox{$\Sigma$} è un insieme \underline{finito} di elementi (lettere, simbolio caratteri).
\\
Sia \mbox{$\Sigma=\{\sigma_0 ... \sigma_k\}$} alfabeto di \begin{math}k\end{math} simboli con cardinalità \mbox{$|\Sigma|=k$}.
\\
\\
Esempi:

\mbox{$\Sigma=\{a, b, ..., z\}$} alfabeto delle lettere romane minuscole

\mbox{$\Sigma=\{0, 1, ..., 9\}$} alfabeto delle cifre arabe

\mbox{$\Sigma=\{0, 1\}$} alfabeto binario
\\
\\
Una \textit{parola} (o stringa) su un alfabeto è una sequenza (finita) di simboli dell'alfabeto.
Esempio: \mbox{$\Sigma = \{a, c, s\}, w = casa$} parola su \mbox{$\Sigma$}
\\
\\
Un \textit{linguaggio} formale è un insieme di parole su un alfabeto.
\\
Esempio:

Insieme di tutti i programmi sintatticamente validi scritti in C++

Insieme di nomi di variabili in Java
\\
\\
La \textit{cardinalità di un linguaggio} finito è il numero di parole che esso contiene.
Esempi:

\mbox{$L_1=\{a, b, ab\}, |L_1|=3$}

\mbox{$L_2=\{aba, abb\}, |L_2|=2$}
\\
\\
Un linguaggio è \textit{finito} se contiene un numero finito di parole.
\\
Un linguaggio è \textit{non finito} se contiene un numero infinito di parole.


\section{Parole}

La lunghezza di una parola \mbox{$w$} è definita con \mbox{$|w|$}. Esempio: \mbox{$|ab| = 2$}.
\\
\\
La parola vuota si indica con \mbox{$\epsilon$}. Essa è l'elemento neutro dell'insieme delle parole ed è l'unica
parola con lunghezza pari a \mbox{$0$}. Quindi vale \mbox{$|\epsilon|=0$}.
\\
\\
Il \textit{numero di occorrenze} del simbolo \mbox{$\sigma$} in una parola \mbox{$w$} è definito con \mbox{$|w|_\sigma$}. Esempio: \mbox{$|aab|_a=2$}.


\subsection{Uguaglianza tra parole}

Sia
\begin{math}
a = a_1...a_k
\end{math}
e
\begin{math}
b = b_1...b_h
\end{math}
con
\begin{math}
k, h \in \mathbf{N},
a_1, ..., a_k, b_1, ..., b_h \in \Sigma
\end{math}.
\\
Allora:
\\
\begin{math}
a=b
\Leftrightarrow
\begin{cases}
k = h \\
\forall i \in \{1, ..., k\} : a_i=b_i
\end{cases}
\end{math}

\newpage

\subsection{Concatenazione di parole}

Sia
\begin{math}
a = a_1...a_k
\end{math}
e
\begin{math}
b = b_1...b_h
\end{math}
con
\begin{math}
k, h \in \mathbf{N},
a_1, ..., a_k, b_1, ..., b_h \in \Sigma
\end{math}.
\\
Allora:
\\
\begin{math}
ab = a_1...a_kb_1...b_h
\end{math}
\\
Questa operazione si chiama concatelazione (o giustapposizione) di parole.
\\
\\
Si noti che \mbox{$\epsilon w = w \epsilon = w$} e che \mbox{$\epsilon \epsilon = \epsilon$}.
\\
\\
L'operazione di concatenazione non gode della proprietà commutativa, infatti con \mbox{$a=anna, b=maria$} si ha
che \mbox{$ab = annamaria \neq mariaanna = ba$}.


\subsubsection{Proprietà associativa della concatenazione}

L'operazione di concatenazione gode invece della proprietà associativa.
Siano \mbox{$a=a_1...a_m, b=b_1...b_n, c=c_1...c_l$} con
\mbox{$a_1, ..., a_m, b_1, ..., b_n, c_1, ..., c_l \in \Sigma, m, n, l \in \mathbf{N}$}
\\
Vale che:
\\
\mbox{$(ab)c = (a_1...a_mb_1...b_n)c_1...c_l = a_1...a_mb_1...b_nc_1...c_l = a_1...a_m(b_1...b_nc_1...c_l) = a(bc)$}


\subsubsection{Stabilità della lunghezza di parole rispetto alla concatenazione}

Siano \mbox{$a=a_1...a_m, b=b_1...b_n$} con
\mbox{$a_1, ..., a_m, b_1, ..., b_n \in \Sigma, m, n \in \mathbf{N}$}
\\
Vale che:
\\
\mbox{$|ab| = |a| + |b| = m + n$}
\\
\\
Dimostrazione diretta:
\\
\mbox{$|ab| = |a_1...a_mb_1...b_n| = |a_1...a_m| + |b_1...b_n| = m + n = |a| + |b|$}
\\
\\
Dimostrazione induttiva:
\\
Passo base:

\mbox{$|ab|=|a_1...a_mb_1...b_n|$}.
\\
Passo induttivo:

Sia
\mbox{$|wv|=|w|+|v|$}
con
\mbox{$w, v \in \Sigma^*, a, b \in \Sigma$}

allora

\mbox{$|wavb| = |wvab| = |wv| + 2 = |w| + |v| + 2 = |w| + |a| + |v| + |b| = |wa| + |vb|$}


\subsection{Prefissi, suffissi e sottostringhe}

...


\section*{Conclusioni}

...

\end{document}
